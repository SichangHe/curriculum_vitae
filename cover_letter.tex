\documentclass[letterpaper,12pt]{article}
\usepackage[margin=1in,bottom=1.5in,footskip=0.5in]{geometry}
\usepackage{enumitem}
\usepackage[hidelinks]{hyperref}

\usepackage{color}
\definecolor{light}{rgb}{0.25, 0.25, 0.25}
\def\light#1{{\color{light}#1}}

\usepackage{lastpage}
\usepackage{fancyhdr}
\fancyhf{}
\renewcommand{\headrulewidth}{0pt}
\fancyfoot[L]{\light Sichang He (Steven) -- University of Southern California}
\fancyfoot[R]{\href{mailto:sichangh@usc.edu}{\light sichangh@usc.edu}}
\pagestyle{fancy}

\usepackage[maxbibnames=99,refsection=section]{biblatex}

\setlength{\parindent}{0pt}  % No indentation
\setlength{\parskip}{1em}    % Double paragraph spacing

\begin{document}

Hello Cloudflare Research Team,
\\

I am a first-year CS PhD student aspiring to measure and
enhance user-facing aspects of the Web. When I first met Vasileios,
I shared my ongoing research on the Coalition for Content Provenance and
Authenticity (C2PA), which we believed was a promising direction to
combat generative AI on challenges like deepfake.
To my great surprise, Vasileios later told me that
Cloudflare is exploring the area around C2PA.
I was thrilled to learn about this internship opportunity,
a perfect alignment with my research interests that
fits my technical background.

When I first met Vasileios at the Internet Measurement Conference,
I was presenting my paper on RPSLyzer.
In interdomain routing, operators of
autonomous systems often document their routing policies to
help coordinate routing, engineer traffic, and troubleshoot routes, in
a language called the Routing Policy Specification Language (RPSL).
These documentations help avoid misconfiguration and attacks that
could cause Internet outage for millions of users.
However, the complete semantics of the RPSL is complicated and
there lacked an accessible comprehensive tool to analyze these policies.
I developed RPSLyzer to fill this gap.
RPSLyzer correctly interprets 99.99\% of RPSL policies, and can check if
BGP routes comply with these policies.
After optimizing RPSLyzer with parallel Rust, I was able to
check 779 millions routes against the policies of
78,701 autonomous systems within 3 hours, revealing RPSL usage patterns and
common mistakes.
Developing RPSLyzer made me realize I can improve the Internet for
end users through research; this inspiration motivated me to pursue a PhD.

Now, this same inspiration motivates me to join Cloudflare's endeavor to
combat challenges brought by generative AI.
Generative AI can potentially flood the Web with
high-quality fake content, which can influence democracy, mislead the public, and
erode trust.
C2PA shows promise in combating this threat, yet its success would depend on
large-scale adoption and answers to various attack vectors.
With Cloudflare's planet scale, outstanding openness in research collaboration, and
a mission to build a better Internet, you are uniquely suitable to
tackle these challenges.
Integrating C2PA into Cloudflare Image was an encouraging step, and
what Vasileios shared with me about your ongoing research in
temper-proof,
hard-to-remove watermarking is towards addressing the metadata removal attack.
I would be thrilled to contribute to these efforts at Cloudflare to
defend trust on the Web.
\\

Thank you for your consideration!\\
Sichang He (Steven)

\end{document}
