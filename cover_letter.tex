\documentclass[letterpaper,12pt]{article}
\usepackage[margin=1in,bottom=1.5in,footskip=0.5in]{geometry}
\usepackage{enumitem}
\usepackage[hidelinks]{hyperref}

\usepackage{color}
\definecolor{light}{rgb}{0.25, 0.25, 0.25}
\def\light#1{{\color{light}#1}}

\usepackage{lastpage}
\usepackage{fancyhdr}
\fancyhf{}
\renewcommand{\headrulewidth}{0pt}
\fancyfoot[L]{\light Sichang He (Steven) -- University of Southern California}
\fancyfoot[R]{\href{mailto:sichangh@usc.edu}{\light sichangh@usc.edu}}
\pagestyle{fancy}

\usepackage[maxbibnames=99,refsection=section]{biblatex}

\setlength{\parindent}{0pt}  % No indentation
\setlength{\parskip}{1em}    % Double paragraph spacing

\begin{document}

Hello Cloudflare Research Team,
\\

I am a first-year CS PhD student aspiring to measure and
enhance user-facing aspects of the Web. Last November, I shared with
Vasileios my ongoing research on the Coalition for Content Provenance and
Authenticity (C2PA), which we believed was promising for
combating challenges like deepfake.
To my surprise, Vasileios later told me that
Cloudflare was exploring this area.
I was thrilled to learn about this internship opportunity,
\textit{a perfect alignment with my research interests that
fits my background in systems development}.

When I met Vasileios at the Internet Measurement Conference,
I was presenting my paper on RPSLyzer from my undergrad.
In interdomain routing,
network operators often publicly document their routing policies using a
standard called the Routing Policy Specification Language (RPSL).
RPSL policies help coordinate routing, engineer traffic, and
troubleshoot routes, thus they help avoid misconfiguration and attacks that
could cause Internet outages for millions of users.
However, the complete semantics of the RPSL is complicated and
there was no accessible comprehensive tool to analyze these policies.
\textit{I developed RPSLyzer to fill this gap.}
RPSLyzer correctly interprets 99.99\% of RPSL policies, and can check if
BGP routes comply with these policies.
I developed RPSLyzer in parallelized Rust; after profile-guided optimization,
I was able to check 779 million routes against the policies of
78,701 autonomous systems within 3 hours, revealing RPSL usage patterns and
common mistakes at a large scale.
\textit{Developing RPSLyzer made me realize I can improve the Internet for
end users through research}; this inspiration motivated me to pursue a PhD.

Now, this same inspiration motivates me to join Cloudflare's endeavor to
combat challenges brought by generative AI.
Generative AI can potentially flood the Web with high-quality fake content,
which can influence democracy, mislead the public, and erode trust.
C2PA shows promise in combating these threats, yet its success would depend on
large-scale adoption and answers to various attack vectors.
\textit{With Cloudflare's planet scale, outstanding openness in
research collaboration, and a mission to build a better Internet,
you are uniquely suitable to tackle these challenges.}
Integrating C2PA into Cloudflare Image was an encouraging step, and
what Vasileios shared with me about your ongoing research in tamper-proof,
hard-to-remove watermarking is towards addressing the metadata removal attack.
I would be thrilled to contribute to these efforts at Cloudflare to
improve content authenticity on the Web.
\\

Thank you for your consideration!\\
Sichang He (Steven)

\end{document}
