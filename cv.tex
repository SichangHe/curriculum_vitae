% Based on https://github.com/fizixmastr

\documentclass[letterpaper,11pt]{article}
\usepackage[margin=0.5in,bottom=1in,footskip=0.5in]{geometry}
\usepackage{titlesec}
\usepackage{enumitem}
\usepackage[hidelinks]{hyperref}

\usepackage{color}
\definecolor{light}{rgb}{0.25, 0.25, 0.25}
\def\light#1{{\color{light}#1}}

\usepackage{lastpage}
\usepackage{fancyhdr}
\fancyhf{}
\renewcommand{\headrulewidth}{0pt}
\fancyfoot[L]{\light{Sichang He (Steven) - University of Southern California}}
\fancyfoot[R]{\light{Page} \thepage{} \light{of \pageref{LastPage}}}
\pagestyle{fancy}

\usepackage[maxbibnames=99,refsection=section]{biblatex}
\addbibresource{cv.bib}
% Make my name bold: https://tex.stackexchange.com/a/327046/294006
\newcommand*{\boldname}[3]{
    \def\lastname{#1}
    \def\firstname{#2}
    \def\firstinit{#3}
}
\renewcommand{\mkbibnamegiven}[1]{
    \ifboolexpr{
        (
            test {\ifdefequal{\firstname}{\namepartgiven}} or
            test {\ifdefequal{\firstinit}{\namepartgiven}}
        ) and test {\ifdefequal{\lastname}{\namepartfamily}}
    }{\mkbibbold{#1}}{#1}
}
\renewcommand{\mkbibnamefamily}[1]{
    \ifboolexpr{
        (
            test {\ifdefequal{\firstname}{\namepartgiven}} or
            test {\ifdefequal{\firstinit}{\namepartgiven}}
        ) and test {\ifdefequal{\lastname}{\namepartfamily}}
    }{\mkbibbold{#1}}{#1}
}
\boldname{He}{Sichang}{S.}

\setlength{\tabcolsep}{0cm}

% Sections formatting
\titleformat{\section}{
  \vspace{-4pt}\scshape\raggedright\large
}{}{0em}{}[\titlerule \vspace{-5pt}]

\newcommand{\CVItem}[1]{
  \item{\small
    {#1 \vspace{-2pt}}
  }
}

\newcommand{\CVSubheadingFrame}[1]{
  \vspace{-2pt}\item
    \begin{tabular*}{0.97\textwidth}[t]{l@{\extracolsep{\fill}}r}
        #1
    \end{tabular*}\vspace{-7pt}
}

\newcommand{\CVSubheading}[4]{\CVSubheadingFrame
    {\textbf{#1} & #2 \\
        \small#3 & #4 \\
    }}

\newcommand{\CVSubHeadingListStart}{\begin{itemize}[leftmargin=0.5cm, label={}]}
\newcommand{\CVSubHeadingListEnd}{\end{itemize}}
\newcommand{\CVItemListStart}{\begin{itemize}}
\newcommand{\CVItemListEnd}{\end{itemize}\vspace{-5pt}}

\newcommand{\uhref}[2]{\href{#1}{\underline{#2}}}

\begin{document}

\noindent
\textbf{\Huge Sichang He (Steven)}\hfill
\uhref{mailto:stevensichanghe@gmail.com}
{\small stevensichanghe@gmail.com}\\[3pt]
Ph.D. Student in Computer Science\hfill
{\small
    \uhref{https://github.com/SichangHe}{GitHub} \&
    \uhref{https://www.youtube.com/@sichanghe}{YouTube} @SichangHe
}\\
University of Southern California\hfill
{\small \uhref{https://sichanghe.github.io/}{Website}}

\section{Education}
\CVSubHeadingListStart

\CVSubheadingFrame
{\textbf{University of Southern California (USC)}&
    Aug. 2024 -- present.\quad Los Angeles, USA\\
    \small Ph.D. student in Computer Science&
    Advisor: Dr. Harsha V. Madhyastha\\
}
\CVItemListStart
\CVItem
{Research focus: User-facing enhancements of the Web.
}
\CVItemListEnd

\CVSubheadingFrame
{\textbf{Duke Kunshan University (DKU) \& Duke University Dual Degree}&
    Aug. 2020 -- May 2024\\
    \small B.S. in Data Science (by DKU) \&&
    Kunshan, China \&\\
    \small \uhref{https://github.com/SichangHe/sichanghe.github.io/files/15386186/duke_diploma.pdf}
    {B.S. in Interdisciplinary Studies} (Subplan: Data Science, by Duke)&
    Durham, NC
}
\CVItemListStart
\CVItem
{In addition to studies in Data Science,
    completed many courses in Computer Science and Mathematics,
    effectively emulating a triple major.
}
\CVItem
{Selected courses with projects:
    \begin{itemize}
        \item Computer Network Architecture (A+):
        async \uhref{https://github.com/SichangHe/CS311}{TCP server and
            client in
            Elixir and CLI REPL shell in Rust}.

        \item Data Acquisition and Visualization (A):
        \uhref{https://github.com/SichangHe/STATS401_final_project}{Poster
            using D3.js, Svelte, and TypeScript}.

        \item Computer Organization and Programming (A+, at Duke):
        Binary search tree in MIPS assembly and ring buffer in C;
        fifth place in PizzaCalc assembly length optimization competition
        (68 lines).

        \item Probability, Random Variables and Stochastic Processes (A):
        \uhref{https://github.com/SichangHe/STATS210_final_project}{PageRank
            DKU website} as a Markov chain.

        \item Numerical Analysis (A+); Intro Abstract Algebra (at Duke),
        Complex Analysis (A).
    \end{itemize}
}
\CVItem
{Achieved full grade on all five courses at Duke University in
    the Fall 2022 semester.
}
\CVItemListEnd

\CVSubheadingFrame
{\textbf{Zhixin High School}&
    Aug. 2017 -- Jun. 2020\quad Guangzhou, China\\
}
\CVItemListStart
\CVItem
{Chemistry Olympiad competition team; Biology Olympiad competition team.
}
\CVItemListEnd

\CVSubHeadingListEnd

\begin{refsection}
    \nocite{he2024rpslyzer}
    \nocite{he2024fedkit}
    \printbibliography[title=Conference Publication]
\end{refsection}

\section{Research Experience}
\CVSubHeadingListStart
\CVSubheading
{Independent Researcher on \uhref
    {https://github.com/SichangHe/internet_route_verification}
    {Internet Route Verification}
}{Apr. 2023 -- present}
{Independent research, Federal University of Minas Gerais, Brazil (Remote)}
{Supervisor: Dr. Italo Cunha}
\CVItemListStart
\CVItem
{Designed and implemented an efficient and comprehensive parser for
    the Routing Policy Specification Language (RPSL),
    a language used in the Internet Route Registry to
    document public inter-domain routing policies,
    guide public peering, and aid routing issue troubleshooting.
    \begin{itemize}
        \item Studied and complied with RPSL semantics in RFCs,
        covering over 99\% of all real-world RPSL use cases.
        \item Optimized the parser to generate detailed verification reports
        for tens of millions of routes within an hour.
        \item Leveraged abstract algebra and Rust language features to
        guarantee correct branching in report generation.
    \end{itemize}
}
\CVItem
{Employed the RPSL parser to verify observed inter-domain routes;
    analyzed verification reports and identified common RPSL usage patterns and
    usage mistakes.
    \begin{itemize}
        \item Provided tooling to verify routes against the RPSL,
        helping improve inter-domain routing security.
        \item Identified and implemented checks for 9 potential reasons why
        routes fail to match the relevant RPSL.
        % \item Obtained detailed statistics of RPSL usage pattern;
        %     derived suggestions to improve RPSL usage.
        \item Conducted interactive data analysis in Rust with the Rust Evcxr REPL.
    \end{itemize}
}
% TODO: Interdomain routing (BGP, policies, RFCs), network datasets (PeeringDB, AS relationship, IRR, …)
% TODO: Try to expand the networking/ lexing/ RPSL semantics.
\CVItemListEnd

\CVSubheading
{Research Assistant for Mobile Federated Learning (FL) Project}
{Mar. 2023 -- May. 2024}
{\uhref{https://github.com/FedCampus}{The FedCampus Team},
    EdgeIntelligence Lab, DKU
}{Supervisor: Dr. Bing Luo}
\CVItemListStart
\CVItem
{Authored \uhref{https://github.com/FedCampus/FedKit}{FedKit},
    open-source SDKs to streamline real-world FL experiments across Android and
    iOS devices,
    enabling training shared ML models collaboratively without
    sharing private data on smartphones.
    \begin{itemize}
        \item Developed a pipeline to convert, train natively,
        and aggregate the same ML models across Android and iOS.
        \item Implemented cross-platform continuous model delivery and training,
        enabling iterative FL design in production.
        \item Contributed to the Flower FL framework:
        revamped the Android example;
        helped correct the iOS example.
        \item Implemented on-device ML training with
        the experimental TensorFlow Lite and
        the proprietary Core ML.
    \end{itemize}
}
\CVItem
{Led and managed the systems development for the FedCampus Android/iOS app,
    leveraging personal health data from over 100 participants to
    conduct real-world FL and federated analytics experiments on
    DKU campus.
    \begin{itemize}
        \item Supervised and mentored four undergrads and
        one M. Eng. student in mobile and web development.
        \item Interviewed and recruited a UI programmer and a designer.
        \item Investigated and led core technology adoption,
        including Flower, Kotlin, and Flutter,
        facilitating development.
    \end{itemize}
}
\CVItemListEnd

\CVSubheading
{Research Assistant for Search Engine Research Project}{Dec. 2021 -- May 2023}
{The Search So Team, DKU}{Supervisor: Dr. Jiang Long}
\CVItemListStart
\CVItem
{Developed a feature-rich \uhref{https://github.com/SichangHe/scraper}
    {open source web scraper} in async Rust to scrape DKU sites,
    intranet, and Duke sites.
}
\CVItem
{Improved backend HTML processing, frontend interface, and version control.
}
\CVItemListEnd
\CVSubHeadingListEnd

\begin{refsection}
    \nocite{luo2024fedcampus}
    \printbibliography[title=Presentation]
\end{refsection}

\section{Teaching Experience}
\CVSubHeadingListStart
\CVSubheading
{Teaching Assistant for COMPSCI 201, DKU}{Nov. 2021 -- Mar. 2022}
{Introduction to Programming and Data Structures}
{Instructor: Dr. Jiang Long}
\CVItemListStart
\CVItem
{Hosted weekly lab sessions and office hours.}
\CVItem
{Created text-based tutorials and the first
    \uhref{https://www.youtube.com/watch?v=yiL-ULPBkvE}
    {video tutorial on development environment setup}.
}
\CVItemListEnd

\CVSubheadingFrame
{\textbf{Math \& CompSci. Tutor, Academic Resource Center, DKU}&
    May 2021 -- May 2022
}
\CVItemListStart
\CVItem
{Tutored dozens of students in MATH 201 - Multivariate Calculus,
    MATH 105 - Calculus,
    and COMPSCI 201.
}
\CVItem
{Obtained \uhref
    {https://github.com/SichangHe/curriculum_vitae/files/11665403/CRLA_certificate.pdf}
    {CRLA's International Tutor Training Program Certification,
        Level I
    }.}
\CVItemListEnd
\CVSubHeadingListEnd

\section{Awards}
\begin{itemize}
    \item \uhref{https://github.com/SichangHe/curriculum_vitae/files/15110729/scholar_athelete_award_DKU_running_club.pdf}
    {Senior Scholar-Athlete Award} through the Running Club,
    DKU Athletics (Apr. 2024)

    \item Silver Medal, International Genetically Engineered Machine (iGEM)
    2022 DKU Team (Oct. 2022)\\{\small
    Developed \uhref{https://github.com/SichangHe/igem-2022-dku-backup}
    {the team wiki} independently; helped non-technical members adopt Git;
    validated protein designs with modeling software.
    }
    \vspace{-4pt}
    \item Dean's List (Spring 2021) \&
    Dean's List with Distinction (Fall 2021, Fall 2022, Spring 2023),
    DKU
    \vspace{-4pt}
    \item Chancellor's Scholarship \& UGRD Entrance Scholarship, DKU
    (merit-based, Fall 2020 -- Spring 2024)
    \vspace{-4pt}
\end{itemize}

\section{Side Projects}
\CVSubHeadingListStart
\CVSubheading
{Open Source Developer \& Maintainer of
    \uhref{https://github.com/lzanini/mdbook-katex}{mdBook-KaTeX}
}{Nov. 2022 -- present}
{Math expression preprocessor for
    \uhref{https://github.com/rust-lang/mdBook}{mdBook} written in Rust;
    over 60,000+ downloads on crates.io
}{GitHub}
\CVItemListStart
\CVItem
{\uhref{https://github.com/lzanini/mdbook-katex/issues/37}
    {Took over maintainership by publishing a fork} when
    it was unmaintained.
}
\CVItem
{Fixed numerous bugs and developed new features,
    resolving more than 20 GitHub issues others had opened.
    \begin{itemize}
        \item Fixed the GitHub CIs for MUSL Linux and Windows builds.
        \item Fixed persisting CommonMark Markdown rendering bugs and
        error handling bugs.
        \item Added support to include math expressions source,
        enable MathML for accessibility, and use custom delimiters.
    \end{itemize}
}
\CVItem
{Improved speed by over 10 times by adopting parallelism and
    avoiding repeated rendering.
}
\CVItem
{Deprecated the problematic static CSS feature gradually and provided an
    alternative.
}
\CVItemListEnd

\CVSubheadingFrame
{\textbf{Author of Open Source \uhref{https://github.com/SichangHe/rails_forum}
        {Web Forum} Using Ruby on Rails}&
    Jun. 2022 -- Aug. 2022
}
\CVItemListStart
\CVItem
{Featured infinitely nested comments;
    deployed on Heroku.
}
\CVItemListEnd
\CVSubHeadingListEnd

\section{Outreach}
\begin{itemize}
    \item Chinese Editor \& Translator at DKU Intersections Journal\hfill
    Jun. 2021 -- Aug. 2021\\{\small
    Co-translated three English articles into Chinese;
    reviewed and edited multiple articles and
    \uhref{https://sites.duke.edu/intersections/}{Intersections' website}.
    }
    \vspace{-4pt}
    \item Active DKU Running Club and Badminton Club member\hfill
    Fall 2020 -- Spring 2024
    \vspace{-4pt}
\end{itemize}

\section{Skills}
\begin{itemize}[leftmargin=0.5cm, label={}]
    \item \textbf{Natural Language}\quad
    Mandarin (native), Cantonese (fluent).\vspace{-4pt}
    \item \textbf{Programming Language}\vspace{-4pt}
    \begin{itemize}\small
        \item Invested and proficient: Rust, Python, Elixir, JavaScript.
        \item Used in projects:
        Kotlin, Dart, Swift, Ruby, Java, Lua, TypeScript,
        Svelte, HTML, CSS, C, SQL, Racket, \LaTeX{}.
        \item Familiar: C++, Fish, Bash, Julia, Go, VimScript, Elisp,
        Mathematica, MATLAB, Scala, R.\vspace{-4pt}
    \end{itemize}
    \item \textbf{Computer Software}\vspace{-4pt}
    \begin{itemize}\small
        % TODO: List other good stuff.
        \item Selected frameworks/libraries:
        Django, Phoenix Framework, Rayon, Tokio, PyParsing, Tailwindcss, D3.js.
        \item Rich experience in setting up and computing on remote servers via
        SSH, Tmux, and Neovim.
        % TODO: Drop the following two and add better stuff.
        \item Custom \uhref{https://github.com/SichangHe/.config}
        {terminal and keyboard-centric development environment}
        with Neovim, tiling window manager Yabai, etc.
        \item Operating systems installation and exploration:
        Windows 10 \& 11, Arch Linux, FreeBSD, QEMU/KVM, Hackintosh.
    \end{itemize}
\end{itemize}
\end{document}
